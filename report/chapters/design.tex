\documentclass[../main.tex]{subfiles}

\begin{document}
\pagebreak
\section{Design}

	\subsection{Global design}

	\subsection{Detailed design}
		\subsubsection{Level Rendering}
			\paragraph{GameView (\tiny client/view/GameView.java\normalsize)}
			The rendering of the level with all it's entities is done in the GameView class. GameView extends the standard Android View, therefore it has to be assigned to a layout. This is done within the GameActivity class (\tiny client/activities/GameActivity.java: onCreate\normalsize). After it is assigned to a layout the init() function is called to initialize all objects used in rendering of the GameView, using the dimensions of the layout. Then two functions are used to render levels: updateLevel and onDraw. \texttt{updateLevel(Level)} prepares a level to be drawn by converting it's tiles and entities to RenderItem's, and figuring out in which order to draw them. \texttt{onDraw(Canvas)} then takes these RenderItem's and draws them to the canvas, in such a way that the player is centered. Locks are used between these functions because onDraw is executed on a different thread compared to updateLevel and they both use the renderItems list.

			\paragraph{RenderItem (\tiny client/view/RenderItem.java\normalsize)} All aspects of rendering a single texture are handled in the RenderItem class. A RenderItem contains the following attributes: \texttt{texture} defines which bitmap to draw; \texttt{x,y} defines the position where the texture should be drawn; \texttt{refX,refY} defines the position within the bitmap used for alignment; \texttt{layer} defines the layer on which this object is drawn; \texttt{animationOffset} the number of frames this object's animation is ahead of the default animation; \texttt{rotation} defines the number of degrees to turn this image counter-clockwise; \texttt{flip} defines whether this image should be flipped horizontally.\\
			The function \texttt{compareTo(RenderItem)} determines which of two RenderItems should be drawn first. The function \texttt{renderTo(Canvas)} renders this RenderItem to the canvas, applying all transformations specified.

			\paragraph{Texture (\tiny client/view/Texture.java\normalsize)} This class handles the loading of textures, to prevent a Bitmap to be loaded every time it needs to be drawn. It has a HashMap that stores all textures using a String as a key. The key corresponds to the path of the texture within the assets/sprites folder. \texttt{getTexture} is a factory method that loads a Bitmap to the HashMap if necessary and returns the corresponding Texture. \texttt{getBitmap} returns the Bitmap associated with a Texture object.

			\paragraph{Animation (\tiny client/view/Animation.java\normalsize)} Animation is a decoration of the Texture class, in order to be able to handle animations. It's \texttt{getBitmap} function takes the current time into account to return a certain frame of the animation.

			\paragraph{Generation of RenderItems} RenderItems are created in the classes of the objects they visualize (\tiny shared/entities/\dots; shared/tiles/\dots\normalsize). This was done to prevent shadow classes for each entity that only generates RenderItems. Unfortunately the objects are shared between the server and client, which means that a server can't be run without compiling all client stuff with it. We didn't find it worthwhile to fix this before the deadline, as we would run the server from within Android Studio for testing. 
	\subsection{Design justification}

\end{document}