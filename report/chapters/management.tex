\documentclass[../main.tex]{subfiles}

\begin{document}
\pagebreak
\section{Project management}
\subsection*{Introduction}
This section contains a log for the process of builing our teams second app for the \textit{Research and Development} Course at Radboud University.
Our process wil be presented in a weekly description of the active tasks that week, project meetings, assignment of tasks, problems we encountered, etc.

\subsection*{Week 19}
After the spring break, in which we did not yet start our new project as most would've been unable to work on it, we decided that a meeting on Monday morning would be wise. This meeting took place in Mercator I, lasted from 10:30 till 12:15, and was attended by Jelmer, Robert, Steven, Thijs and halfway through also by Thomas.
The main topic of this meeting was choosing which app idea we would run with. We had three prominent ones:
\begin{itemize}
	\item A multiplayer dungeon crawler. This idea had us very excited at first, but we realized that this would mean a very similar project structure to the Sokoban app of the previous assignment, so we dismissed the idea.
	\item An medical application in cooperation with medicine students, as suggested by Patrick van Bommel. This idea seemed cool, but did not motivate us as much, as we would have to deal with outside requirements instead of our own ideas about what would make a good addition to the app.
	\item A mario-party-like game with multiplayer minigames. By now we had settled on the idea of building a game, as we had a lot of fun doing th eprevious assignment. By making a game centred around minigames, we think we'll be able to make this process fun fr ourselves. We'll have to put some effort in making a connection between two phones.
\end{itemize}

The third is the idea we had settled on about halfway through the meeting. We made some sketched of the project layout and brainstormed on some minigames (like spyfall and charades)

We planned our next meeting for Thursday, third block. By then, Steven will have set-up a new Github repository.

\end{document}