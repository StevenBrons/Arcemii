\documentclass[../main.tex]{subfiles}

\begin{document}
\pagebreak
\section{Project management}
\subsection*{Introduction}
This section contains a log for the process of building our teams second app for the \textit{Research and Development} Course at Radboud University.
Our process will be presented in a weekly description of the active tasks that week, project meetings, assignment of tasks, problems we encountered, etc.

\subsection*{Week 19}
After the spring break, in which we did not yet start our new project as most would've been unable to work on it, we decided that a meeting on Monday morning would be wise. This meeting took place in Mercator I, lasted from 10:30 till 12:15, and was attended by Jelmer, Robert, Steven, Thijs and halfway through also by Thomas.
The main topic of this meeting was choosing which app idea we would run with. We had three prominent ones:
\begin{itemize}
	\item A multiplayer dungeon crawler. This idea had us very excited at first, but we realized that this would mean a very similar project structure to the Sokoban app of the previous assignment, so we dismissed the idea.
	\item An medical application in cooperation with medicine students, as suggested by Patrick van Bommel. This idea seemed cool, but did not motivate us as much, as we would have to deal with outside requirements instead of our own ideas about what would make a good addition to the app.
	\item A Mario-party-like game with multiplayer minigames. By now we had settled on the idea of building a game, as we had a lot of fun doing the previous assignment. By making a game centred around minigames, we think we'll be able to make this process fun fr ourselves. We'll have to put some effort in making a connection between two phones.
\end{itemize}

The third is the idea we had settled on about halfway through the meeting. We made some sketched of the project layout and brainstormed on some minigames (like spyfall and charades)

We planned our next meeting for Thursday, third block. By then, Steven will have set-up a new Github repository.
\bigbreak\noindent
Thursday we had our next meeting, which was attended by everyone. In the previous days, we had all thought about the idea of something using a server, and had become sceptical about how good of an idea it was. We discussed the following risks:
\begin{itemize}
	\item Our app will probably be tested by one person, but they will have to be able to use the app's full functionality to properly grade it. Robert has sent an e-mail to our TA with questions about this, so we'll await their response and then look at this risk again.
		\item It is risky that our whole app relies on one connection. If something goes wrong in the connection part, the whole app will suffer from it.
		\item Four weeks for an app is already a short timeframe. Will we be able to afford the time to spend on multiplayer functionality? This is something we'll have to decide upon once we have more concrete ideas about games we'd like to make.
\end{itemize}
Because of the above mentioned risks, we further discussed the possibilities of single-player games, as the idea of the app being a game remained unchanged. Thomas and Thijs were charged with the task to think about games in the rogue-like genre, in order to see if such a game would be a valuable app idea. Steven will make the structure of a client-server program. This way, we can have some extra days to decide on our idea, as we'd like the whole group to be fully behind it, hard as it is, while also getting started on some code.


\subsection*{Week 20}
We had our first meeting of the week on Monday morning, again. Over the weekend, more doubts about the party-game idea had arisen. Everyone was present, though Robert was there only the second half. First order of business was making the final decision for our idea. We all felt this was actually long due, and were all a bit frustrated because of this. So: today we were to make a definite decision and we would stick with it.

Our main two problems with the party game were design-problems: we would have to make some smaller, individually not very impressive things, that would have a very simple structure. This did not feel like enough of a challenge, even the multiplayer aspect taken into account. The second problem was that it would be hard to make it one well-rounded app. Additionally, because it had pieces that were so disjunct from others, we would evade the whole project idea of this course. To conclude: we switched back to the idea of one game. till multiplayer. The remaining of the meeting, we discussed possibilities. Firstly, we looked at possibilities to expand on one of the discussed minigames:
\begin{itemize}
	\item Curve Fever
	\item Bomberman
	\item Spyfall
\end{itemize}
However, for the aforementioned reason of not being complex enough, we looked further. A game like hill-climb racing was discussed, but we did not have concrete enough ideas for implementing multiplayer into that game. In the end, we landed on the idea of a multiplayer rogue-like yet again. Here is a small run-down of how we envision it:

You create a party with your friends. Every player is presented a random selection of `abilities', of which they can choose some. They enter a level of some kind and fight enemies. On the end of a level, there is a `boss fight', after which the players are rewarded and can go down to the next level. Some opportunities for player-versus-player were mentioned, like the strongest player being a final boss after all levels had been completed, but we decided to postpone these ideas.

When Robert joined us for the meeting, he and Steven discussed the way we would tackle the multiplayer/server aspect code-wise. They proposed different approaches:
\begin{itemize}
	\item Steven suggested to start with a full-scale client-server connection, so that we would not have to spend much time later on in the project to change existing code to make it suitable for multiplayer and such.
	\item Robert suggested to first make the game in single-player form and add multiplayer later, as this could prove tricky and time-consuming.
\end{itemize} 
It was agreed upon that Steven would work on his vision for the next meeting, where we would decide if we would stick with it or not.
\bigbreak\noindent
Thursday was our next meeting. Everyone but Steven was present, because he `is too cool to use the bus'. We started a Discord-call with him however, and he showed us his work from the past days. In the end, he used Roberts approach. Thijs made a little character animation to look at possible styles and possibility for animations. We'll definitely use pixel-art, as it is less time-consuming while still being charming.

Main item on the agenda were dividing task:
\begin{itemize}
	\item Jelmer: Graphics rendering
	\item Bram: Being able to create and join parties
	\item Robert: Dungeon generation
	\item Thijs: Sprites
	\item Thomas: Looking at abilities
\end{itemize}

\subsection*{Week 21}
As usual, we had our first meeting on Monday morning. Everyone was present, though Robert only attended the second half. We started with discussing what everyone had achieved over the weekend. Actually, the divided tasks were pretty big, so everyone was still busy with them. Because of Jelmer's rendering, we could see the first animations on screen. 

We then discussed the game envisioned in more detail:
\begin{itemize}
	\item We want players to be able to play the game in short bursts (like in the breaks in the middle of lectures), so we're aiming at 15-20 minutes per total game. That means that we'll probably have 3 levels per game. In the future, we can expand this with an option to choose between a short, medium and long game.
	\item We'll start with making just 3 abilities, of which the player can choose 2 or 3 at the beginning of the game. The idea is that this will be easy to expand on later.
	\item For enemies, we'll start with 3 basic ones:
	\begin{itemize}
		\item An enemy that does damage on touch: a slime. A slime will be able to make a jump attack.
		\item An enemy that attacks with range: a skeleton that will shoot at players.
		\item A boss that is bigger and does more damage.
	\end{itemize}
	\item We discussed abilities being on cooldown versus using mana to use them. The majority is pro cooldown, but we'll come back to this later when the implementation becomes relevant.
\end{itemize}
We set our next meeting for Thursday and gave everyone new tasks (well, most of them are continuations of the last ones). New ones are:
\begin{itemize}
	\item Steven: make it able to test multiplayer.
	\item Thomas: work on ability selection
	\item Bram: finalize the lobby activity.
\end{itemize}
\bigbreak\noindent
Thursday was our second meeting. Everyone but Thomas was present. This is very inconvenient, as he did not finish his task.

Jelmer had continued with his rendering and the only thing left is rendering outside of a level. 

Thijs was assigned the task to work on some sprites, and he made these animations:
\begin{itemize}
	\item Idle and walking animations for 4 player characters
	\item Idle and jumping animations for a slime
	\item Idle, walking and shooting animations for a skeleton
	\item A floating animations for a boss
	\item A tree
\end{itemize}
Robert is almost done with the dungeon generator, and he'll finish this for next Monday. We made some more dungeon specifications for him to generate: enemy locations and start en goal positions.

Thijs and Bram discussed the layout and style for the different activities. Mainly, the lobby activity, where the party waits for the game to start and players choose their abilities, has to change so that abilities will actually be able to be chosen. This is a task given to Bram for next Monday. 

Steven aims to make all game-logic that is not directly dependent on tasks given to others. 

Thomas is given the task to start on the in-game UI. 


\subsection*{Week 22}
At Steven's request, because he had an appointment, the meeting started later than usual. This meant a shorter meeting, as Thomas had to leave us in the break. Sadly, not everyone was on time, so we effectively had only 45 minutes for our meeting. This is very sub-optimal as we start to realize that there is still a \textit{lot} of work to be done before we have an presentable application.

Robert has completed the dungeon generator, but will make some final adjustments before he will merge it with the master branch on GitHub. 

Steven did some work on passing objects between the server and the client. Still a lot has to be done in this part before we can properly test everything. A big part of our struggles are that most tasks seem intertwined with others: Thomas would like to work on in-game abilities, but he can't until a proper connection is made. Bram would like to work on the lobby activity, but this has little meaning when there are no abilities.

An important breakthrough this weekend was that Bram was able to join Jelmer via the app! This means the connection that \textit{is} being made, will probably work correctly later on. Still though, we realize that working with multiplayer was probably a very ambitious idea and the risks we were afraid of in earlier weeks are staring us in the face.

Bram spent most of his weekend working on a bug that causes the player not to enter a good multiplayer mode after they have first switched to singleplayer mode -- as a side note: there is a singleplayer toggle added, which just `tricks' the app into thinking it's local device is the server, for the purpose of testing and later on for true offline game possibilities.

Thomas has not been busy yet with the UI and controls. We urge him to look at this quickly, we want (theoretical) player movement by Thursday!
\bigbreak\noindent
Thursday was Ascension Day, and the university was closed. We did not arrange a meeting, as most had family-obligations to attend, but we discussed some important things via Whatsapp, so that everyone would still be able to work this weekend.

There were some problems with the dungeon generation (namely that rooms are not connected) that Robert fixed immediately at Wednesday.

Thursday, Thomas added a joystick for movement, but this is not yet fully integrated in the app.
\end{document}
