\documentclass[../main.tex]{subfiles}

\begin{document}
\pagebreak
\section{Evaluation}
In this section, we look back at our project and discuss parts of the process we'd like to handle differenlty in the future, possible issues or opportunities for our app, etcetera.

\bigbreak\noindent
Overall, we are satisfied with our product. We think we made something pretty impressive, given we had only four weeks for the project. Multiplayer sounded like an obstacle, too big to overcome. But we did. It feels like we made a game a definitive step up from Sokoban, code-wise at least. 

Although we achieved our techinal goals, Arcemii does not (yet) achieve \textit{all} of our goals. Namely: we wanted the game to be something we'd enjoy playing in breaks, for example. This is possible now, but honestly, we haven't been playing the game as much afterwards as we did with Sokoban. Why is this? It would be easy to say that the game just isn't much fun, but we don't think that's fair. We spend a good amount of time brainstorming about the gameplay, so that can't be it. However, we had a short amount of time to make Arcemii, so definetely not all our envisioned features are already implemented. This creates opportunities for the future though! We structured our code in a way, that it is easily scalable. Now that we have the basis, we can add anything we want. This is our main look on Arcemii's future. Some additions we'd like to make, include:
\begin{itemize}
	\item More enemies and bosses.
	\item More abilities to choose from.
	\item 'Meta progress', where you can permanently unlock new abilities.
	\item Sound effects for getting hit, hitting enemies, etc.
\end{itemize}

There may be some issues that need fixing before these additions should be made. For example: the ip-address to connect to in multiplayer mode is hardcoded in the code. This means that you cannot connect when the server is not active. For the app to become succesful, this should be different (by using a server that's always on for example, one with a bigger capacity, etc).

\bigbreak\noindent
Things we learned and our satisfaction with the development process go hand in hand. Surely, all of us learned something about programming, XML, using Android Studio, making animations\dots The list goes on. But these things were never our main focuspoints when we were in a meeting, for example. The real hurdle was the teamwork. Everyone was confident in eachothers skills, but with Sokoban we learned that not everyone wants to do their tasks at the same time (some have a `if I work on it now, I won't have to later' approach, while others more of a `why do it today when it can be done tomorrow'). Because of this, setting deadlines and such admittedly did not go as smoothly as we'd hoped for. While everything came together in the end, it would have been less stressful for everyone if we could've been clearer about when to do things. This is a big learning point.





\end{document}